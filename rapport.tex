\documentclass[12pt]{article}

\usepackage{amsfonts, amsmath, amssymb, amstext, latexsym}
\usepackage{graphicx, epsfig}
\usepackage[utf8]{inputenc}
\usepackage[french]{babel}
\usepackage{exscale}
\usepackage{amsbsy}
\usepackage{amsopn}
\usepackage{fancyhdr}

\usepackage{amsmath, amssymb, amsfonts, amsthm, fouriernc, mathtools}
% mathtools for: Aboxed (put box on last equation in align envirenment)
\usepackage{microtype} %improves the spacing between words and letters

\usepackage{graphicx}
\graphicspath{ {./pics/} {./eps/}}
\usepackage{epsfig}
\usepackage{epstopdf}


%%%%%%%%%%%%%%%%%%%%%%%%%%%%%%%%%%%%%%%%%%%%%%%%%%
%% COLOR DEFINITIONS
%%%%%%%%%%%%%%%%%%%%%%%%%%%%%%%%%%%%%%%%%%%%%%%%%%
\usepackage[svgnames]{xcolor} % Enabling mixing colors and color's call by 'svgnames'
%%%%%%%%%%%%%%%%%%%%%%%%%%%%%%%%%%%%%%%%%%%%%%%%%%
\definecolor{MyColor1}{rgb}{0.2,0.4,0.6} %mix personal color
\newcommand{\textb}{\color{Black} \usefont{OT1}{lmss}{m}{n}}
\newcommand{\blue}{\color{MyColor1} \usefont{OT1}{lmss}{m}{n}}
\newcommand{\blueb}{\color{MyColor1} \usefont{OT1}{lmss}{b}{n}}
\newcommand{\red}{\color{LightCoral} \usefont{OT1}{lmss}{m}{n}}
\newcommand{\green}{\color{Turquoise} \usefont{OT1}{lmss}{m}{n}}
%%%%%%%%%%%%%%%%%%%%%%%%%%%%%%%%%%%%%%%%%%%%%%%%%%




%%%%%%%%%%%%%%%%%%%%%%%%%%%%%%%%%%%%%%%%%%%%%%%%%%
%% FONTS AND COLORS
%%%%%%%%%%%%%%%%%%%%%%%%%%%%%%%%%%%%%%%%%%%%%%%%%%
%    SECTIONS
%%%%%%%%%%%%%%%%%%%%%%%%%%%%%%%%%%%%%%%%%%%%%%%%%%
\usepackage{titlesec}
\usepackage{sectsty}
%%%%%%%%%%%%%%%%%%%%%%%%
%set section/subsections HEADINGS font and color
\sectionfont{\color{MyColor1}}  % sets colour of sections
\subsectionfont{\color{MyColor1}}  % sets colour of sections

%set section enumerator to arabic number (see footnotes markings alternatives)
\renewcommand\thesection{\arabic{section}.} %define sections numbering
\renewcommand\thesubsection{\thesection\arabic{subsection}} %subsec.num.

%define new section style
\newcommand{\mysection}{
\titleformat{\section} [runin] {\usefont{OT1}{lmss}{b}{n}\color{MyColor1}} 
{\thesection} {3pt} {} } 

%%%%%%%%%%%%%%%%%%%%%%%%%%%%%%%%%%%%%%%%%%%%%%%%%%
%		CAPTIONS
%%%%%%%%%%%%%%%%%%%%%%%%%%%%%%%%%%%%%%%%%%%%%%%%%%
\usepackage{caption}
\usepackage{subcaption}
%%%%%%%%%%%%%%%%%%%%%%%%
\captionsetup[figure]{labelfont={color=Turquoise}}

%%%%%%%%%%%%%%%%%%%%%%%%%%%%%%%%%%%%%%%%%%%%%%%%%%
%		!!!EQUATION (ARRAY) --> USING ALIGN INSTEAD
%%%%%%%%%%%%%%%%%%%%%%%%%%%%%%%%%%%%%%%%%%%%%%%%%%
%using amsmath package to redefine eq. numeration (1.1, 1.2, ...) 
%%%%%%%%%%%%%%%%%%%%%%%%
\renewcommand{\theequation}{\thesection\arabic{equation}}

%set box background to grey in align environment 
\usepackage{etoolbox}% http://ctan.org/pkg/etoolbox
\makeatletter
\patchcmd{\@Aboxed}{\boxed{#1#2}}{\colorbox{black!15}{$#1#2$}}{}{}%
\patchcmd{\@boxed}{\boxed{#1#2}}{\colorbox{black!15}{$#1#2$}}{}{}%
\makeatother
%%%%%%%%%%%%%%%%%%%%%%%%%%%%%%%%%%%%%%%%%%%%%%%%%%




%%%%%%%%%%%%%%%%%%%%%%%%%%%%%%%%%%%%%%%%%%%%%%%%%%
%% DESIGN CIRCUITS
%%%%%%%%%%%%%%%%%%%%%%%%%%%%%%%%%%%%%%%%%%%%%%%%%%
\usepackage[siunitx, american, smartlabels, cute inductors, europeanvoltages]{circuitikz}
%%%%%%%%%%%%%%%%%%%%%%%%%%%%%%%%%%%%%%%%%%%%%%%%%%



\makeatletter
\let\reftagform@=\tagform@
\def\tagform@#1{\maketag@@@{(\ignorespaces\textcolor{red}{#1}\unskip\@@italiccorr)}}
\renewcommand{\eqref}[1]{\textup{\reftagform@{\ref{#1}}}}
\makeatother
\usepackage{hyperref}
\hypersetup{colorlinks=true}

%%%%%%%%%%%%%%%%%%%%%%%%%%%%%%%%%%%%%%%%%%%%%%%%%%
%% PREPARE TITLE
%%%%%%%%%%%%%%%%%%%%%%%%%%%%%%%%%%%%%%%%%%%%%%%%%%
\title{\blue Principes et Méthodes Statistiques \\
\blueb TP 2017}
\author{Pierre Bouvier, Nolwenn Cadic, Maxime Gourgoulhon}
\date{\today}
%%%%%%%%%%%%%%%%%%%%%%%%%%%%%%%%%%%%%%%%%%%%%%%%%%

\newcommand{\noi}{\noindent}
\newcommand{\dsp}{\displaystyle}

\textheight 25cm
\textwidth 16cm
\oddsidemargin 0cm
\evensidemargin 0cm
\topmargin 0cm
\hoffset -0mm
\voffset -20mm


\pagestyle{plain}


\begin{document}

\maketitle
\baselineskip7mm

% \noi ENSIMAG $1^{\footnotesize \mbox{ère}}$ année   \hfill Mars 2017


\vspace{1cm}


\begin{center}
{\Large \bf Principes et Méthodes Statistiques 

\vspace{3mm}

TP 2017

\vspace{3mm}

}
\end{center}

\noi \rule[0.5ex]{\textwidth}{0.1mm}


\section*{Introduction au problème}

Les données considérées sont issues d'une démarche d'analyse de signaux oculométri-ques, c'est-à-dire de signaux de suivi du mouvement des yeux au cours du temps. Ces signaux ont été mesurés sur des participants lors de tâches consistant à lire des textes traitant de thèmes prédéfinis : football, chasse aux oiseaux, physique nucléaire, art contemporain, etc... 
L'objectif est que le lecteur identifie le thème du texte le plus vite possible.

Lors de la lecture, deux types de mouvements d'yeux alternent : la fixation, qui correspond à la lecture proprement dite d'un mot ou deux voire moins, et où les yeux sont quasi-immobiles, et la saccade, qui est un déplacement très rapide portant les yeux vers un autre mot ou une autre partie d'un mot long.

On considère un premier groupe de $n=227$ personnes à qui on a fait lire $n$ textes pris au hasard dans une première série de textes. On mesure pour chaque personne le nombre $X$ de fixations nécessaires pour identifier le type du texte lu. Les données sont fournies dans le fichier {\it groupe1.txt}.
Le fichier {\it groupe2.txt} contient des données similaires pour un second groupe de $m=168$ personnes à qui on a fait lire $m$ textes pris au hasard dans une autre série de textes.

L'objectif du TP est d'analyser ces données et de comparer les 2 groupes. On propose de modéliser la loi de probabilité de $X$ par une loi géométrique ou une loi binomiale négative.


%%%%%%%%%%%%%%%%%%%%%%%%%%%%%%%%%%%%%%%%%%%%%%%%%%%%%%%%%%%%%%%
\section{Analyse d'échantillons de loi binomiale négative}
%%%%%%%%%%%%%%%%%%%%%%%%%%%%%%%%%%%%%%%%%%%%%%%%%%%%%%%%%%%%%%%

Une variable aléatoire $X$ est de loi binomiale négative ${\cal BN}(r,p)$ de paramètres $r \in \mathbb{N}^*$ et $p \in [0, 1]$  si elle est à valeurs dans $ \{r, r+1, ...\}$ et que :
$$P(X=x) = \binom{x-1}{r-1}(1-p)^{x-r}p^r, \, \forall x \in \{r, r+1, ...\}$$

Son espérance est $E[X]={\dsp \frac{r}{p}}$ et sa variance est $Var[X]={\dsp \frac{r(1-p)}{p^2}}$.

En {\tt R}, {\tt rnbinom(n,r,p)} simule un échantillon de taille $n$ d'une variable aléatoire $Y$ telle que $Y+r$ est de loi ${\cal BN}(r,p)$. 

La loi géométrique ${\cal G}(p)$ est la loi ${\cal BN}(1,p)$. En {\tt R}, {\tt rgeom(n,p)} simule un échantillon de taille $n$ d'une variable aléatoire $Y$ telle que $Y+1$ est de loi ${\cal G}(p)$.

On considère un échantillon $X_1, \ldots, X_n$ de variables aléatoires indépendantes et de même loi ${\cal BN}(r,p)$.

Dans un premier temps, on suppose que le paramètre $r$ est connu. 

\vspace{3mm}

\begin{enumerate}

\renewcommand{\labelenumi}{\arabic{section}.\arabic{enumi}.}

\item Estimer $p$ par la méthode des moments, puis par la méthode de maximum de vraisemblance. Montrer que les deux estimateurs sont égaux. On notera $\hat{p}_n$ cet estimateur.

\vspace{3mm}

\item Appliquer le théorème central-limite à $\bar{X}_n$. En déduire un intervalle de confiance bilatéral asymptotique de seuil $\alpha$ pour $p$.

\vspace{3mm}

\item Dans le cas de la loi géométrique ($r=1$), construire le graphe de probabilités avec la méthode usuelle basée sur la fonction de répartition. En déduire une estimation graphique $p_{g_1}$ de $p$. Evaluer la qualité de cette méthode sur la base de jeux de données simulées.

\vspace{3mm}

\item Dans le cas de la loi binomiale négative avec $r>1$, cette méthode ne fonctionne pas car la fonction de répartition n'a pas d'expression simple. On propose alors une méthode similaire basée sur un autre critère.

\begin{enumerate}

\vspace{3mm}

\item Quand $X$ est de loi ${\cal BN}(r,p)$, calculer ${\dsp \frac{P(X=x)}{P(X=x+1)}}$. 

\vspace{3mm}

\item Proposer une méthode permettant d'évaluer à partir d'un nuage de points si la loi Binomiale Négative est un modèle plausible. En déduire deux estimations graphiques de $p$, $p_{g_2}$ et $p_{g_3}$.

\vspace{3mm}

\item Evaluer la qualité de cette méthode sur la base de jeux de données simulées.

\end{enumerate}

\vspace{3mm}

\hspace{-5mm} Dans un deuxième temps, on considère que les 2 paramètres $p$ et $r$ sont inconnus. 

\vspace{3mm}

\item Calculer les estimateurs de $p$ et $r$ par la méthode des moments, $\tilde{p}_n$ et $\tilde{r}_n$.

\vspace{3mm}

\item Proposer une méthode numérique permettant de calculer les estimateurs de maximum de vraisemblance $\hat{p}_n$ et $\hat{r}_n$.

\end{enumerate}


%%%%%%%%%%%%%%%%%%%%%%%%%%%%%%%%%%%%%%%%%%%%%%%%%%%%%%%%%%%%%%%
\section{Analyse des deux jeux de données}
%%%%%%%%%%%%%%%%%%%%%%%%%%%%%%%%%%%%%%%%%%%%%%%%%%%%%%%%%%%%%%%

\begin{enumerate}

\renewcommand{\labelenumi}{\arabic{section}.\arabic{enumi}.}

\item Effectuer une analyse de statistique descriptive des 2 jeux de données, incluant des représentations graphiques et des calculs d'indicateurs statistiques. Commenter les résultats.

\vspace{3mm}

\item Les lois géométrique et binomiale négative sont-elles des modèles plausibles pour ces deux jeux de données ? Pour chaque jeu de données, choisir un modèle pertinent, estimer ses paramètres de toutes les façons possibles et, quand c'est possible, donner un intervalle de confiance asymptotique de seuil $\alpha=5\%$ pour ces paramètres.

\vspace{3mm}

\item Pour les deux jeux de données, estimer le nombre moyen de fixations nécessaires pour identifier le type du texte lu et estimer la probabilité que ce nombre soit supérieur à 10. Quelle conclusion générale tirer de cette analyse ?

\end{enumerate}


%%%%%%%%%%%%%%%%%%%%%%%%%%%%%%%%%%%%%%%%%%%%%%%%%%%%%%%%%%%%%%%
\section{Vérifications expérimentales à base de simulations}
%%%%%%%%%%%%%%%%%%%%%%%%%%%%%%%%%%%%%%%%%%%%%%%%%%%%%%%%%%%%%%%


\begin{enumerate}

\renewcommand{\labelenumi}{\arabic{section}.\arabic{enumi}.}

\item Choisir $p$, $n$, $m$ et $\alpha$. Simuler $m$ d'échantillons de taille $n$ de la loi ${\cal G}(p)$. Pour chacun d'entre eux, calculer l'intervalle de confiance asymptotique de seuil $\alpha$ pour $p$ obtenu dans la question 2.2. Comparer la proportion d'intervalle contenant la vraie valeur de $p$ à $1-\alpha$. Quel est l'impact du choix des valeurs de $p$, $n$, $m$ et $\alpha$ ?

\vspace{3mm}

\item Vérification de la loi faible des grands nombres. Simuler $m$ échantillons de taille $n$ de la loi ${\cal G}(p)$. Calculer le nombre de fois où l'écart en valeur absolue entre la moyenne empirique et l'espérance de la loi simulée est supérieure à un $\epsilon$ à choisir. Faire varier $n$ et conclure.

\vspace{3mm}

\item Vérification du théorème central-limite. Simuler $m$ échantillons de taille $n$ de la loi ${\cal G}(p)$. Sur l'échantillon des $m$ moyennes empiriques, tracer un histogramme et un graphe de probabilités pour la loi normale. Faire varier $n$ en partant de $n=5$ et conclure.

\vspace{3mm}

\item Choisir $r$, $p$, $n$ et $m$. Simuler $m$ échantillons de taille $n$ de la loi ${\cal BN}(r,p)$. Pour chaque échantillon, calculer les estimateurs des moments de $r$ et $p$, $\tilde{r}_n$ et $\tilde{p}_n$. Utilisez ces simulations pour évaluer le biais et l'erreur quadratique moyenne de ces estimateurs. Faire varier $n$ et commenter les résultats.

\end{enumerate}

\end{document}